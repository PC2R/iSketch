\documentclass[a4paper, 11pt]{article}
\usepackage[utf8]{inputenc}
\usepackage[T1]{fontenc}
\usepackage{lmodern}
\usepackage[french]{babel}
\usepackage{fullpage}
\usepackage{graphicx}

\title{Devoir de programmation PC2R : iSketch}
\author{Clara \bsc{Muller} \& Théo \bsc{Lebourg}}
\date{}

\begin{document}

\maketitle

\section{Côté serveur}
Pour développer la partie serveur de iSketch, nous avons opté pour le
langage OCaml. Nous allons dans un premier temps expliquer quels ont
été nos choix pour organiser notre serveur avant de s’attarder un peu
plus en détail sur le code.

\subsection {Création du serveur}
Lorsque l'on lance l’executable, la première étape du programme est la
création du serveur (pour qu’il soit prêt à recevoir les connexions
des futurs clients) ainsi que l’initialisation de variables qui seront
utilisées une fois le jeu lancé.

\bigskip La création de notre objet serveur reprend les étapes
classiques vues en cours. On crée tout d’abord \textbf{une prise} à
l’aide de l’appel système \verb+Unix.socket+ (en utilisant le domaine
Internet, le flot d’octets et le protocole de communication par
défaut), puis on récupère \textbf{une adresse dans le domaine
  internet} (à l’aide de l’adresse Internet de la machine locale et
d’un numéro de port défini par l’utilisateur) afin de pouvoir associer
les deux. Ensuite, on rend la prise capable d’accepter les
connexions. Enfin, on met en place une boucle infinie qui accepte les
demandes de connexions et qui va créer un objet joueur à chaque
nouvelle demande (nous verrons plus tard comment nous gérons le nombre
limite de joueurs par partie).

\bigskip Juste avant de commencer à accepter les demande de connexions
des joueurs, on initialise quelques variables :
\begin{itemize}
\item Un tableau de taille le nombre de joueurs maximal pour une
  partie et dont chaque case contiendra une string avec le
  \textit{pseudo} du joueur ainsi que son nombre de \textit{points} au
  cours du jeu sous la forme \textit{pseudo/points/}
\item Un tableau contenant l’ensemble des mots présent dans le fichier
  \textit{dictionnaire} donné par l’utilisateur et dans lequel on
  tirera au hasard un mot à chaque tour
\end{itemize}


\section{Côté client}

\end{document}
