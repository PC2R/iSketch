\documentclass[a4paper, 11pt]{article}
\usepackage[utf8]{inputenc}
\usepackage[T1]{fontenc}
\usepackage{lmodern}
\usepackage[french]{babel}
\usepackage{fullpage}
\usepackage{graphicx}

\title{Devoir de programmation PC2R : iSketch}
\author{Clara \bsc{Muller} \& Théo \bsc{Lebourg}}
\date{}

\begin{document}

\maketitle

\section{Côté serveur}
Pour développer la partie serveur de iSketch, nous avons opté pour le
langage OCaml. Nous allons dans un premier temps expliquer quels ont
été nos choix pour organiser notre serveur avant de s’attarder un peu
plus en détail sur le code.

\subsection {Lancement du serveur et de la partie}
Lorsque l'exécutable est lancé avec les paramètres qui vont bien, la
première étape du programme est la création du serveur (pour qu’il
soit prêt à recevoir les connexions des futurs clients) ainsi que
l’initialisation de certaines variables qui seront utilisées une fois
le jeu lancé.

\bigskip La création de notre objet serveur reprend les étapes
classiques vues en cours : on crée une \textbf{une prise} (en
utilisant le domaine Internet, le flot d’octets et le protocole de
communication par défaut) et on récupère \textbf{une adresse dans le
  domaine internet} (à l’aide de l’adresse Internet de la machine
locale et d’un numéro de port défini par l’utilisateur) afin de
pouvoir associer les deux, puis on rend la prise capable d’accepter
les connexions et enfin, on met en place une boucle infinie qui
accepte les demandes de connexions et qui va créer un thread joueur à
chaque nouvelle demande (nous verrons plus tard comment nous gérons le
nombre limite de joueurs par partie).

Notons qu’avant de commencer à accepter les demandes de connexion,
nous lisons le fichier dictionnaire et nous stockons dans une liste
chacun des mots présent dans le fichier. Nous avons opté pour cette
solution pour éviter de devoir ouvrir et fermer le dictionnaire à
chaque tour de round pour piocher un mot. 

\bigskip En parallèle des demandes de connexions des joueurs, on lance
un thread qui va commencer par attendre que tous les joueurs soient
connectés. Une fois cette condition remplie, le premier round
s’exécute : un mot est choisi (et retiré de la liste pour éviter de le
tirer plus tard) ainsi qu’un dessinateur et ces informations sont
envoyées à tous les joueurs via la commande \verb+NEW_ROUND+. Ensuite,
le serveur attend cette fois-ci qu’un mot soit trouvé et une fois
cette condition remplie, il envoie à tous les joueurs les commandes
\verb+END_ROUND+ et \verb+SCORE_ROUND+ qui donnent respectivement le
nom du vainqueur et le mot qu’il fallait trouver et le score de tous
les joueurs. Enfin, un nouveau round peut commencer en suivant le même
schéma qui vient d’être décrit.

\subsection {Connexion d’un joueur}

Plusieurs possibilités sont offertes au joueur pour accéder au jeu :
\bigskip
\begin{itemize}
\item Avec la commande \verb+CONNECT/user/+ : le serveur vérifie alors
  que le nom \textit{user} n’a pas déjà été choisi (en vérifiant la
  base de données des comptes utilisateurs ainsi que les joueurs déjà
  connectés à la partie en cours) et crée un nouveau nom si ce n’est
  pas le cas (en concaténant un nombre à la fin de \textit{user}).
\item Avec la commande \verb+REGISTER/user/password/+ : le serveur
  vérifie alors que le nom \textit{user} n’a pas déjà été choisi et
  l’ajoute à la base de données si c’est le cas.
\item Avec la commande \verb+LOGIN/user/password/+ : le serveur
  vérifie alors que les noms \textit{user} et \textit{password} sont
  corrects et s’ils le sont, le serveur vérifie également que le
  joueur n’est pas déjà en train de jouer.
\end{itemize}
\bigskip Enfin, notons qu’une commande \verb+ACCESSDENIED/+ est
envoyée ou bien si le nombre maximum de joueurs pour une partie est
atteint ou bien si les noms et mots de passe ne correspondent pas ou
bien si un joueur tente de se connecter alors qu’il est déjà connecté.

\subsubsection {Quand un mot est trouvé par un joueur}

Voici le déroulement général des étapes effectuées par le serveur
lorsqu’un joueur a trouvé le mot que le dessinateur était en train de
dessiner. Notons que puisque les joueurs évoluent chacun sur un
thread, nous avons protégé cette série d’opérations par un
\textbf{mutex}.

\begin{enumerate}
\item Le serveur se charge d’envoyer la commande
  \verb+WORD_FOUND/joueur/+ à tous les joueurs de la partie.
\item Le serveur attribue le nombre de points qui va bien au joueur
  qui a trouvé le mot
\item Un timeout est lancé si et seulement si le joueur en question
  est le premier à trouver le mot et il reste des joueurs connectés
  (sinon, s’il est le seul, alors un nouveau round peut commencer si
  la partie n’est pas terminée)
\end{enumerate}

\subsection {Extensions}

\subsubsection{Discussion instantanée}

La mise en place de la discussion instantanée côté serveur n’a pas
été trop compliquée étant donné qu’il suffit de récupérer la chaîne de
caractère contenue dans la commande \verb+TALK+ pour la renvoyer à
tous les joueurs via la commande \verb+LISTEN+. Par ailleurs, comme
suggéré dans l’énoncé, nous avons utilisé la commande
\verb+BROADCAST+ pour annoncer à tous les joueurs qu’un acte de triche
avait été signalé par l’un d’entre eux.

\subsubsection {Comptes utilisateurs}

Côté serveur, nous avons opté pour le stockage des noms et mots de
passe des joueurs dans un simple fichier texte. Cependant, avant
d'être stocké sur ce fichier, le mot de passe est salé (dynamiquement
pour éviter les attaques par tables arc-en-ciel) puis hashé avec la
fonction de hashage MD5 (cette fonction de hachage n'est certes plus
fiable depuis longtemps mais nous n’avons pas trouvé dans la librairie
standard d'OCaml d’autres fonctions plus performantes telle SHA-256).
Ainsi, à chaque inscription, on ajoute une ligne contenant le nom du
joueur, le mot de passé hashé et salé et le sel.

\bigskip Par ailleurs, notons que pour l’instant, les mots de passe
circulent en clair lorsqu’ils sont envoyés depuis le client.

\subsubsection {Serveur HTTP de statistiques}

Afin de rendre les statistiques de iSketch (nombre de parties gagnées
et perdues pour les joueurs enregistrés dans la base de données)
disponibles via un navigateur internet, nous avons repris la base de
notre serveur acceptant les connexions des joueurs mais en fixant
cette fois-ci le port 2092 comme énoncé dans le sujet. Il nous a
ensuite fallu trouver le moyen de répondre aux requêtes \verb+GET+ du
navigateur envoyées lorsque le joueur entre l’URL
\textit{adresse\_du\_serveur:2092}. Pour cela, lorsque nous recevons
une requête \verb+GET+ (de la forme \textit{GET / HTTP/protocole})
nous commençons par récupérer le protocole pour construire notre
requête qui sera construite de la manière suivante : une première
ligne de la forme \textit{HTTP/protocole 200 OK}, suivie de quelques
headers, une ligne vide et enfin notre page HTML contenant les
informations des joueurs.

\bigskip Par ailleurs, nous avons muni la page HTML d’une balise qui
permet le rafraîchissement de la page toutes les minutes pour ainsi
actualiser les statistiques.

\section{Côté client}

\subsection {L'interface graphique}

Le langage JAVA a semblé le plus adapté pour gérer l'interface graphique
avec l'utilisateur. En effet, les packages swing, et  awt permettent une
grande variétés de composant simple à appréhender.
La fenetre principale est donc divisé en 4 zones :
 - en haut à droite, la zone de dessin, active ou non selon le rôle du joueur durant le round
 - en bas à droite, une zone de chat qui fonctionne en permanence pour tous les joueurs
 - à gauche, la majeur partie de l’écran correspond au fenêtre de dialogues
entre le client et le serveur : on peut y lire la liste des joueurs et les différentes actions de la partie en cours
 - en bas à gauche, un champ pour envoyer des proposition de mots
   au serveur. Ce champ n'est actif que lorsque le joueur n'est pas le dessinateur du round.

L'interface graphique transmet les informations entre les différentes
hiérarchie de classes jusqu'au classes principales qui font envoyer
et recevoir les information du serveur.

\subsection{Les classes principales}

La principale classe du Client n'est pas \textit{Client.java} comme
l'on pourrait le supposer.
Néanmoins elle contient des éléments essentiels comme : la
fonction main du programme, la lecture des arguments du programme
et l'initialisation de la connexion à l'aide de la socket avec la bonne adresse
et le bon port. Une fois la socket crée, elle peut lancer un canal de
lecture et un canal d'écriture tous les deux lié au serveur qui vont
être donné à la classe qui va gérer tous les messages : \textit{Messenger.java}
Tous les envois et réception de message transitent par cette classe où
ils sont traité et envoyé au serveur si c'est le destinataire, ou vers
la zone de l'interface graphique appropriée. 
Pour cela la classe \textit{MainWindow.java} qui contient tous les composant graphiques
traitent également transmet quasiment tous les messages entre l'interface et
la classe \textit{Messenger.java}. Elle fait remonter toutes les actions de
l'utilisateur vers la classe \textit{Messenger.java} et redistribue toutes les
informations du serveur vers les composants graphiques concernés.
Mais cette la classe MainWindow est crée uniquement si le joueur à
réussi à se connecter correctement. Pour cela, \textit{Messenger.java} propose
un choix au joueur sur la manière de se connecter (anonymement, en
s'enregistrant ou en utilisant son identifiant et son mot de passe),
puis écrit sur le canal d'écriture du serveur la commande connexion
correspondante. Elle attend ensuite sur le canal de lecture une réponse.
Si cette réponse commence bien par \verb+WELCOME+ ce qui signifie que, la
connexion a réussi, l'interface graphique se lance ainsi que le
système d'écoute et d'envoi de messages. Si n'importe quelle autre
réponse en récupérée, nous considérons que la connexion a échouée,
aucun autre objet n'est créé et la classe \textit{Client.java} referme la socket.


\subsection{Le système d'écoute et de d'envoi de message au serveur}

Le client ayant moins de demandes en parallèle à traiter que le serveur,
nous avons pus traiter la principale difficulté avec deux thread
lancé tous les deux en même temps dont les rôle respectifs sont d'écouter
toutes les commandes du serveur pour l'un et d'envoyer toute les commandes
au serveur pour l'autre.
Pour que le thread d'envoi de message ne soit actif que lorsqu'un message
doit être envoyé, on le le fait s'endormir sur une variable déclarée dans
\textit{Messenger.java} qui lorsqu'elle est modifiée, notifie le thread concerné à
l'aide de la méthode notify(), qui l'envoie alors au serveur avant de se
rendormir.
Le thread de réception des messages, lui, lit en permanence sur le canal
de lecture du serveur et renvoie l'interprétation de la commande à la
classe \textit{Messenger.java} lorsqu'il en reçoit une par l'intermédiaire d'une
variable dans lequel il a le droit d'écrire.
Ces deux variables sont de type \textit{Message.java} et permettent des actions
simple comme écrire un message ou récupérer le message présent.
Pour ne pas que les commandes se chevauchent et que des informations soient
perdues, toutes les actions des thread, d'écriture ou de lecture dans
les variable sont effectuées dans des blocs synchronized ce qui empêche
tout écrasement d'information avant qu'elle ne soit traitée.

\subsection{Le système de dessin}

Lorsque le joueur est le dessinateur du round, la zone de dessin de
l'interface devient active. Les lignes simples ont été implémentées avec
plusieurs tailles et plusieurs couleur différentes. Pour tracer un trait,
l'utilisateur clique sur la zone de dessin d'où il souhaite faire commencer
l’extrémité de son trait et relâche la souris à l'endroit souhaité de l'autre
extrémité du trait.
Grâce à l'interface \textit{MouseListener.java} implémentée par la classe \textit{BoardPanel.java},
nous pouvons récupérer les deux point distinct pour tracer un trait entre
les deux, un point étant distingué par sa position, sa  couleur et sa taille.
Pour pouvoir dessiner plusieurs traits, chaque point est stocké dans une liste
qui lorsque nous faisons appel à la méthode repaint() retrace les anciens traits
ainsi que les nouveaux qui auraient pu être ajoutés.
Pour détecter les changements de couleurs ou de taille sélectionnés par l'utilisateur,
on a utilisé des classe implémentant l'interface \textit{ActionListener.java}.
Ainsi lorsque le joueur trace un trait, effectue un changement de taille
ou de couleur, l'action en renvoyée jusqu'à la classe \textit{Messenger.java} qui va
transmettre l'information au serveur à l'aide de la commande correspondante.

Si le joueur n'est pas dessinateur, cette zone n'est pas active mais, il reçoit toutes les informations du serveur qui lui permettent de redessiner sur sa propre zone tous les traits tracé par le dessinateur.

\end{document}
